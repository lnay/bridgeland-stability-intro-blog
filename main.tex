\documentclass[]{article}
\usepackage{projpreamble}


\begin{document}
\newcommand{\RR}{\mathbb{R}}
\newcommand{\ZZ}{\mathbb{Z}}
\newcommand{\CC}{\mathbb{C}}
\newcommand{\triangcat}{\mathcal{T}}
\newcommand{\aut}{\mathrm{Aut}}
\newcommand{\stab}{\mathrm{Stab}}
\newcommand{\PSL}{\mathrm{PSL}}
\newcommand{\gltworplustilde}{\widetilde{\mathrm{GL}}(2,\RR)^+}

\maketitle
\tableofcontents

\section{What this is}

Pure mathematics material, especially what is studied by graduate students, suffers from a pedagogical issue...
in the sense that research papers are not tuned for pedagogy at all.
Research is about proving results first and foremost, and the most direct choice of proof or construction often
involves skipping the context and reasoning that led to it.
This is an issue for graduate students who often only have such a paper to learn about the subject,
contrasting hard with the undergraduate experience, where teaching content has had time to mature.

This isn't to say that authors do not try, the opening example in the foundational paper defining Bridgeland
stability conditions \cite{bridgeland2006stabilityconditionstriangulatedcategories} was a motivating context
demonstrating the classical ``$\mu$-stability'' on smooth projective curves as a Bridgeland stability condition.
However it's quite brief and easy to miss if not too familiar with the subject.
This post fleshes this opening example out, and can be read as a primer for the technical development of the
theory in the rest of the paper.
The value is in seeing what the goal is first, before diving into the technicalities,
which puts you more in line with the author's mindset
(they did not pursue this abstraction out of nowhere).

\section{Bridgeland vs $\mu$ stabilities - differences}

\begin{flexbox}
\begin{definition}[$\mu$-stability]
Given a Coherent sheaf
\end{definition}
\begin{definition}[Bridgeland-stability]
Given a triangulated category
\end{definition}
\end{flexbox}

\printbibliography

\end{document}
